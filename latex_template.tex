% do NOT change these next two lines
\documentclass[11pt]{article}
\usepackage[margin=1in]{geometry}

% Edit this to represent your group members, title, and date
\author{Jane Doe, Jean le Rond d'Alembert, Emmy Noether}
\date{\today}
\title{A Proof of the Smoothness of Solutions to the Navier-Stokes 
Equations}

% Environments for theorems/definitions/lemmas/examples if needed
\newtheorem{theorem}{Theorem}
\newtheorem{definition}{Definition}
\newtheorem{lemma}{Lemma}
\newtheorem{example}{Example}

% package for making dummy, filler text (can delete if you want)
\usepackage{lipsum}

\begin{document}
\maketitle

%% None of the text below is strictly needed. Included just to illustrate

\lipsum[1]

\begin{definition}[Definition of Definition]
An example of how definitions can be made. \label{myDefinition}
\end{definition}

\lipsum[1]

\begin{example}
The approach for examples, lemmas, theorems is similar.\label{myExample}
\end{example}

\LaTeX~even lets you reference Definition~\ref{myDefinition} and 
Example~\ref{myExample} in a way that is adaptive to any re-ordering you 
make.


You can include your bibliography/works-cited in any format you desire, 
here's an example using in-line BibTeX (this is fine for a short report, if 
you plan on using \LaTeX~extensively in the future, it's worthwhile to learn 
how to correctly utilize \texttt{.bib} files). This (also) lets you 
reference bibliography items by keyword and not have to worry about the 
order changing (adding a citation causing reference 2 to become reference 3 
etc. is handled by the \TeX~engine, \cite{texbook}).

\begin{thebibliography}{9}
\bibitem{texbook}
Donald E. Knuth (1986) \emph{The \TeX{} Book}, Addison-Wesley Professional.

\bibitem{lamport94}
Leslie Lamport (1994) \emph{\LaTeX: a document preparation system}, Addison
Wesley, Massachusetts, 2nd ed.
\end{thebibliography}

\end{document}
